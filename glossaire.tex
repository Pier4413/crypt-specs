\newglossaryentry{json}
{%
	name={json}, % le terme à référencer (l'entrée qui apparaitra dans le glossaire)
	text={JSON},
	description={JavaScript Object Notation (JSON) est un format de données textuelles dérivé de la notation des objets du langage JavaScript. Il permet de représenter de l’information structurée comme le permet XML par exemple. Créé par Douglas Crockford entre 2002 et 20051, la première norme du JSON est ECMA-404 qui a été publiée en octobre 20032, il est actuellement décrit par les deux normes en concurrence : RFC 82593 de l’IETF et ECMA-4044 de l'ECMA \cite{JSONWikipedia}}, % la description du terme (sans retour à la ligne)
	sort={json}, % si le mot contient des caractère spéciaux, ils ne seront pas pris en compte
	plural={JavaScript Object Notation (JSON)} % la forme plurielle du terme
}

\newglossaryentry{aes}
{
	name={aes}, % le terme à référencer (l'entrée qui apparaitra dans le glossaire)
	text={AES},
	description={Advanced Encryption Standard (AES) est un algorithme de chiffrement symétrique, aujourd'hui considéré comme le plus sûr du monde et le plus utilisé \cite{AESWikipedia}}, % la description du terme (sans retour à la ligne)
	sort={aes}, % si le mot contient des caractère spéciaux, ils ne seront pas pris en compte
	plural={Advanced Encryption Standard (AES)} % la forme plurielle du terme
}